\documentclass[paper=a4, fontsize=11pt]{scrartcl}
\usepackage[svgnames]{xcolor}
\usepackage[a4paper,pdftex]{geometry}

\usepackage[german]{babel}
\usepackage[utf8x]{inputenc}
\usepackage[T1]{fontenc}
\usepackage{lmodern}
\usepackage[numbers]{natbib}

\usepackage{fullpage}
\usepackage{fancyhdr}

\usepackage{float}
\restylefloat{table}
\restylefloat{figure}
\usepackage{tabularx}
\usepackage{paralist}
\usepackage{bytefield}

\pagestyle{fancy}
	\fancyhead[L]{INSO - Industrial Software\\ \small{Institut für Rechnergestützte Automation | Fakultät für Informatik | Technische Universität Wien}}
	\fancyhead[C]{}
	\fancyhead[R]{}
	\fancyfoot[L]{SMS senden via AT-Befehle}
	\fancyfoot[C]{}
	\fancyfoot[R]{Seite \thepage}
	\renewcommand{\headrulewidth}{1pt}
	\renewcommand{\footrulewidth}{1pt}
\setlength{\headheight}{0.5cm}
\setlength{\headsep}{0.75cm}

\usepackage{hyperref}
	\hypersetup{
		colorlinks,
		citecolor=black,
		filecolor=black,
		linkcolor=black,
		urlcolor=black
	}

\usepackage{pdfpages}

\usepackage[babel,german=quotes]{csquotes}

% ------------------------------------------------------------------------------
% Title Setup
% ------------------------------------------------------------------------------
\newcommand{\HRule}[1]{\rule{\linewidth}{#1}} % Horizontal rule

\makeatletter               % Title
\def\printtitle{%
	{\centering \@title\par}}
\makeatother

\makeatletter               % Author
\def\printauthor{%
	{\centering \normalsize \@author}}
\makeatother


\title{	\Large SMS senden via AT-Befehle% Subtitle of the document
	\\[2.0cm] \HRule{0.5pt} \\
	\LARGE \textbf{\uppercase{Dokumentation}} % Title
	\HRule{2pt} \\[1.5cm]
	\normalsize Wien am \today \\
	\normalsize Technische Universität Wien \\[1.0cm]
	\normalsize ausgeführt im Rahmen der Lehrveranstaltung \\
	\LARGE 183.286 – (Mobile) Network Service Applications
}


\author{
	Gruppe 35 \\[2.0cm]
	Stefan Gamerith \\
	0925081 \\
	E 066 937 \\
	Software Engineering and Internet Computing \\[2.0cm]
}

\begin{document}
% ------------------------------------------------------------------------------
% Title Page
% ------------------------------------------------------------------------------
\thispagestyle{empty} % Remove page numbering on this page
\printtitle
	\vfill
\printauthor
\newpage
% ------------------------------------------------------------------------------
% Kurzfassung
% ------------------------------------------------------------------------------
\section*{Kurzfassung}
Im Rahmen der Lehrveranstaltung \textit{(Mobile) Network Service Applications} wurde eine 
Java Applikation entwickelt welche es ermöglicht, SMS-Nachrichten beliebiger Länge
zu versenden. Der Nachrichtentext wird bei der Übertragung mit dem GSM 7-Bit 
Alphabet~\cite{gsm-7-bit} enkodiert. 
% ------------------------------------------------------------------------------
% Table of Contents
% ------------------------------------------------------------------------------
\newpage
\tableofcontents
\newpage
% ------------------------------------------------------------------------------
% List of Tables
% ------------------------------------------------------------------------------
\newpage
\listoftables
\listoffigures
\newpage
% ------------------------------------------------------------------------------
% Document
% ------------------------------------------------------------------------------
\section{Einleitung}
Aufbauend der Aufgabenstellung \textit{Entwicklung einer Applikation zum
Senden von SMS} wurde eine Java-Applikation entwickelt welche es ermöglicht, 
SMS-Nachrichten beliebiger Länge zu versenden. Das Programm bedient sich
sogenannter AT-Befehle~\cite{at-befehle} zur Kommunikation mit dem Mobiltelefon.
Die Datenübertragung geschieht über eine serielle Schnittstelle.

\section{Problemstellung/Zielsetzung}
Die Java-Applikation kommuniziert mit dem Handy (Nokia 6212) über die serielle
Schnittstelle. Intern werden SMS-Nachrichten unter Verwendung sogenannter AT-Befehle
versendet.
\subsection{AT-Befehle}~\\
AT-Befehle werden in 
	\begin{inparaenum}[1)]
		\item \textit{Basic commands}, 
		\item \textit{Extended commands}, 
		\item \textit{Parameter type commands} und  
		\item \textit{Action type commands}
	\end{inparaenum} unterteilt \cite{at-befehle}. 
Während \textit{Basic commands} auf jedem Modem verfügbar sind, kann die Verfügbarkeit
bzw. die genaue Syntax der \textit{Extended commands} mit nachgestelltem Fragezeichen
abgefragt werden. Letztere verfügen zusätzlich noch über den Präfix \texttt{AT+}. 
Ein erfolgreich ausgeführter Befehl wird vom Modem mit \texttt{OK} quittiert, andernfalls
schickt es \texttt{ERROR} und bricht die Abarbeitung weiterer Befehle ab.

Tabelle~\ref{tab:at-befehle} listet alle notwendigen AT-Befehle zum erfolgreichen
Versenden von SMS-Nachrichten\\

\begin{table}[H]
	\begin{tabularx}{\textwidth}{l|X}
		\textbf{AT-Befehl} & \textbf{Beschreibung} \\
		\hline
		\texttt{ATZ} & Befehl zum Zurücksetzten des Modems.\\
		\texttt{AT+CMGF} & Definiert, welches Format für die SMS-Übertragung verwendet wird. (\texttt{0} für \textit{PDU-Mode} und \texttt{1} für \textit{Text-Mode})\\
		\texttt{AT+CMGS} & Versendet direkt eine SMS. Die genaue Syntax ist abhängig vom Übertragungs-Mode. \\
	\end{tabularx}
	\caption{Menge aller notwendigen AT-Befehle zum Versenden von SMS-Nachrichten}
	\label{tab:at-befehle}
\end{table}
\subsection{PDU Mode}~\\
Eine wichtige Anforderung für die Implementierung war das Versenden der SMS-Nachrichten
im PDU-Mode. Dieser ermöglicht die Übertragung von Binärdaten welche als hexadezimale Zeichenkette
angegeben werden können.

Die zwei wichtigsten Nachrichtentypen im PDU-Mode sind SUBMIT-Nachrichten und DELIVER-Nachrichten.
Abbildung~\ref{fig:pdu} zeigt schematisch den Aufbau einer SMS-SUBMIT-Nachricht im PDU-Mode~\cite{pdu-mode} (in Byte Blöcken).
\begin{figure}[H]
	\begin{bytefield}[bitwidth=\textwidth/8]{8}
		\bitbox{8}{SMSC information (0x00)}\\
		\bitbox{8}{Erstes Octet der SMS-SUBMIT Nachricht (0x01)}\\
		\bitbox{8}{TP-Message Reference (0x00)}\\
		\bitbox{8}{Länge der Empfänger Adresse}\\
		\bitbox{8}{Type of Address (0x91)}\\
		\wordbox[lrt]{1}{Telefonnummer des Empfängers (im Semi-Oktetten)} \\
		\skippedwords \\
		\wordbox[lrb]{1}{} \\
		\bitbox{8}{Protocol Identifier (0x00)}\\
		\bitbox{8}{Data Coding scheme (0x00)}\\
		\bitbox{8}{Payload Länge (in Septetten)}\\
		\wordbox[lrt]{3}{Payload} \\
		\skippedwords \\
		\wordbox[lrb]{1}{} \\
	\end{bytefield}
	\caption{Aufbau einer SMS-SUBMIT-Nachricht im PDU-Mode}
	\label{fig:pdu}
\end{figure}

Um auch Nachrichten mit einer Länge größer als 160 Zeichen versenden zu können, 
wurde ein eigener Nachrichtentyp spezifiziert. Der Unterschied zum
herkömmlichen Nachrichtentyp ist der sogenannte \textit{User Data Header} welcher am
Beginn des Payloads zu finden ist. Abbildung \ref{fig:gsm-udh} zeight schematisch den
Aufbau des User Data Headers.\\
\begin{figure}[H]
	\begin{bytefield}[bitwidth=\textwidth/8]{8}
		\bitbox{8}{Länge des User Data Headers (Anzahl der Oktetten)} \\
		\bitbox{8}{User Data Header Identifier (0x08)} \\
		\bitbox{8}{Länge der Referenz Nummer (0x04)} \\
		\wordbox[lrt]{2}{Referenz Nummer (0x0000)} \\
		\bitbox{8}{Anzahl der Teile} \\
		\bitbox{8}{Teilnummer} \\
	\end{bytefield}
	\caption{Schematischer Aufbau des User Data Headers}
	\label{fig:gsm-udh}
\end{figure}

\subsection{Text Kodierung}~\\
Eine weitere Anforderung war die Verwendung des GSM 7-Bit Alphabets~\cite{gsm-7-bit} für die Textkodierung. Diese ermöglicht
die Übertragung von 128 unterschiedlichen ASCII-Zeichen. Nach der Umwandlung der Zeichen
in Septetten (7-Bit Blöcke) müssen diese noch mit einem geeigneten Algorithmus in Oktetten
(8-Bit Blöcke) gepackt werden, da Daten üblicherweise byteweise übertragen werden. 

Abbildung~\ref{fig:gsm-encoding} veranschaulicht grafisch die Konvertierung von 8 Septetten in
7 Oktetten.\\
\begin{figure}[H]
\begin{bytefield}[bitwidth=\textwidth/56]{56}
	\bitheader{0,6,7,13,14,20,21,27,28,34,35,41,42,48,49,55}\\
	\bitbox{7}{Septet 1} & \bitbox{7}{Septet 2} & \bitbox{7}{Septet 3} & \bitbox{7}{Septet 4} & \bitbox{7}{Septet 5} & \bitbox{7}{Septet 6} & \bitbox{7}{Septet 7} & \bitbox{7}{Septet 8}\\
	\bitbox{8}{Oktet 1} & \bitbox{8}{Oktet 2} & \bitbox{8}{Oktet 3} & \bitbox{8}{Oktet 4} & \bitbox{8}{Oktet 5} & \bitbox{8}{Oktet 6} & \bitbox{8}{Oktet 7}\\
	\bitheader{0,7,8,15,16,23,24,31,32,39,40,47,48,55}\\
\end{bytefield}
	\caption{Umwandlung von 8 Septetten in 7 Oktetten}
	\label{fig:gsm-encoding}
\end{figure}
Falls die Anzahl der zu übertragenden Septetten nicht einem Vielfachen von 8 entspricht,
müssen sogenannte \textit{Paddingbits} noch am Schluss angefügt werden.

\section{Methodisches Vorgehen - Fallbeispiel}
\subsection{Laufzeitanforderungen}~\\
Da Nokia nur Windows Treiber zur seriellen Kommunikation mit dem Handy zur Verfügung stellt,
fiel die Entscheidung auf Windows XP SP3 als primäre Entwicklungs- und Laufzeitumgebung.

Eine weitere Vorraussetzung für eine einwandfreie
Funktionsweise ist die Installation des Nokia 6212 NFC SDKs. Letztere installiert einen COM Port, über jenem die gesamte
Kommunikation mit dem Mobiltelefon statt findet. Wichtig ist, das dieser Port AT-Befehle akzeptiert. 
Dazu öffnet man am besten eine Konsole (zum Beispiel \textit{Hyperterminal}) und tippt \texttt{AT} gefolgt von Enter ein. 
Anschließend sollte dieser Befehl mit \texttt{OK} bestätigt werden. Sollte dies nicht der Fall sein, empfiehlt sich 
die Neukonfiguration beziehungsweise die Vergabe eines anderen COM-Ports bei einer Neuinstallation des Modems.

Zur seriellen Kommunikation ist weiters die Installation der \textit{Java Comm API} notwendig. Hierzu muss die native Bibliothek \textit{win32com.dll} in den Order \texttt{\%JAVA\_HOME\%/jre/bin} und die Datei \textit{javax.comm.properties} in den Order \texttt{\%JAVA\_HOME\%/jre/lib} kopiert werden (beide Dateien wurden von der LVA-Leitung zur Verfügung gestellt).

\subsection{Build Management}~\\
Als Buildmanagement Werkzeug wurde Apache Maven~\cite{maven} verwendet. Der Befehl \texttt{mvn package} erstellt ein
ausführbares jar-Archiv im Order \textit{target}. Anschließend genügt  das Ausführen des Befehls \textit{java -jar sms-sender-1.0-SNAPSHOT-jar-with-dependencies.jar}
um das Programm zu starten. 

\subsection{Konfigurationsdateien}~\\
Die primäre Konfigurationsdatei ist \textit{sendsms.properties}.
Abbildung \ref{fig:sms-prop} zeigt beispielhaft eine ebensolche Datei.\\ 

\begin{figure}[H]
	\begin{verbatim}
		csvfile=sendsms.csv
		port=COM4
	\end{verbatim}
	\caption{Beispiel der Konfigurationsdatei \textit{sendsms.properties}}
	\label{fig:sms-prop}
\end{figure}

Daneben muss noch ein Comma Separated File (CSV) existieren (spezifiziert in \textit{sendsms.properties}). 
Abbildung \ref{fig:sms-csv} zeigt beispielhaft die Datei \textit{sendsms.csv}.\\
 \begin{figure}[H]
 	\begin{verbatim}
		+436507112256,This is a short text!
		+436507112256,This is a not so short text!
 	\end{verbatim}
 	\caption{Beispiel der Konfigurationsdatei \textit{sendsms.csv}}
 	\label{fig:sms-csv}
 \end{figure}
 
 \section{Fazit}
Essentiell für die korrekte Funktionsweise dieses Programms ist eine serielle Verbindung zum Mobiltelefon über jene AT-Befehle abgesetzt werden können. 
Sollte dies nicht möglich sein, schafft wahrscheinlich ein Blick in den Gerätemanager Abhilfe. Das Handy sollte als Modem im Gerätemanager gelistet sein. 
Ist dies nicht der Fall, hilft eine Neuinstallation des Handytreibers. Weiters sollte der richtige serielle Port angegeben werden. Dieser kann im Konfigurationsdialog
des Modems im Reiter \texttt{Advanced} und einem Klick auf \texttt{Advanced Port Settings} konfiguriert werden. 

% ------------------------------------------------------------------------------
% Literatur
% ------------------------------------------------------------------------------
\newpage
\bibliography{Literatur}
\bibliographystyle{plainnat}
\end{document}